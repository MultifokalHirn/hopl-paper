\documentclass[oribibl]{llncs}
\usepackage[utf8]{inputenc}
\usepackage{llncsdoc}
\usepackage{url}
\usepackage{amsmath}
\usepackage{amssymb}
\usepackage{graphicx}
%
\author{Lennard Wolf \\
        lennard.wolf@student.hpi.de}
\institute{ Hasso Plattner Institute \\
            Prof.-Dr.-Helmert-Straße 2-3 \\
            14482 Potsdam \\
            Germany}
\title{A Structuralist View on Object-Oriented \\ Thought Paradigms by Employment of CLOS}
\date{\today}
%
\begin{document}
\markboth{A Structuralist View on Object-Oriented \\ Thought Paradigms by Employment of CLOS}{A Structuralist View on Object-Oriented \\ Thought Paradigms by Employment of CLOS}
\thispagestyle{empty}
\vfill

%
\maketitle
%
\begin{abstract}
A Structuralist View on Object-Oriented Thought Paradigms by Employment of CLOS
\end{abstract}
%
\section{Introduction}

% Topic / Domain


% Problem


% Contribution


% Outline
The following is a summary of each section. In section...

\section{Context}
\label{sec:context}
In this Section we will give an overview of the history of CLOS, then we will introduce the ideas behind Lisp and Object-Orientation in general, and finally we will introduce the three main concepts in CLOS that make it unique.

\subsection{Object-Orientation}
\label{sec:oo}

\begin{itemize}
\item OO = objects + classes + inheritance
\item Objects have attributes (state) and behaviour 
\item These Objects can interact with one another by sending messages
\item Classes are code templates with initial values for state and implementations of behavior
\item Inheritance: deriving more specific classes from others 
\item Most popular class-based: Objects are Instances of Classes
\item Gives programmers the ability to model the real world
\end{itemize}


\subsection{Lisp}
\label{sec:lisp}

\begin{itemize}
\item List Processor: Everything is a list (plus 3 4)
\item No Difference between data and code
\item Use Lisp Macros to easily extend Lisp
\item Therefore it is easy to create Domain Specific
Languages etc.
\item possible to add OO to Lisp in Lisp
\item explain Macros in detail
\end{itemize}


\subsection{History of CLOS}
\label{sec:history}

\begin{itemize}
\item Many implementations (“Tower of Babel” situation) (Flavors, CommonLoops)
\item 1986: Workgroup of different researchers from Xerox PARC (CommonLoops) and Symbolic Inc. (New Flavors) worked on it together 
\item CLOS: System built on top of Common Lisp
\item meta-object protocol consisting of 33 functions \& 8 macros
\item CLOS is portable to different LISP implementations
\end{itemize}

\subsection{Main Concepts}
\label{sec:concepts}

\subsubsection{Generic Functions}
\label{sec:genfun}
blablablablablablablablablablablabla

\subsubsection{Multiple Inheritance}
\label{sec:mulinh}
blablablablablablablablablablablabla

\subsubsection{Method Combination}
\label{sec:metcom}
blablablablablablablablablablablabla

\begin{figure}[ht]
    \centering
    \includegraphics[width=0.6\textwidth]{images/example.jpg}
    \caption{test}
    \label{fig:computing_primes}
\end{figure}

\section{Problem}
\label{sec:problem}

\begin{itemize}
\item how can we build a people description interface that is easily extendable etc., like DSL (?) for people and their positions
\item using langs like Java (?) might be problematic, give example
\end{itemize}

\section{Approach}
\label{sec:approach}


\begin{itemize}
\item 3 concepts will be used
\item maybe give function that lets user add classes
\end{itemize}


\section{Implementation}
\label{sec:implementation}

\begin{itemize}
\item give code details for concepts from Approach
\end{itemize}

\section{Evaluation}
\label{sec:evaluation}

\begin{itemize}
\item DSL type declaration is easily maintainable
\item using langs like Java (?) might be problematic, give example
\end{itemize}


\section{Conclusion}
\label{sec:conclusion}
Write me pl0x

\newpage
\nocite{*}
\bibliographystyle{splncs}
\bibliography{hopl-paper}

\end{document}